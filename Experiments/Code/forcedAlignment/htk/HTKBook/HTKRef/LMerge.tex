%
% HLMBook - V.Valtchev    26/05/98
%
% Updated - Gareth Moore  15/01/02 - 1/3/02
%

\newpage
\mysect{LMerge}{LMerge}

\mysubsect{Function}{LMerge-Function}

\index{lmerge@\htool{LMerge}|(}
This program combines one or more language models to produce an output 
model for a specified vocabulary.  You can only apply it to word
models or the class $n$-gram component of a class model -- that is,
you cannot apply it to full class models.

\mysubsect{Use}{LMerge-Use}

\htool{LMerge} is invoked by typing the command line
\begin{verbatim}
   LMerge [options] wordList inModel outModel
\end{verbatim}
The word map and class map are loaded, word-class mappings performed and 
a new map is saved to \texttt{outMapFile}. The output map's name will be
set to 
\begin{verbatim}
Name = inMapName%%classMapName
\end{verbatim}

The allowable options to \htool{LMerge} are as follows

\begin{optlist}
{
  \ttitem{-f s} Set the output LM file format to \texttt{s}. Available options
  are \texttt{text}, \texttt{bin} or \texttt{ultra} (default \texttt{bin}).

  \ttitem{-i f fn} Interpolate with model \texttt{fn} using weight \texttt{f}.

  \ttitem{-n n} Produce an \texttt{n}-gram model.

}
\end{optlist}
\stdopts{LMerge}


\mysubsect{Tracing}{LMerge-Tracing}

\htool{LMerge} Does not provide any trace options. However, trace 
information is available from the underlying library modules 
\htool{LWMap} and \htool{LCMap} by setting the appropriate trace
configuration parameters.

\index{lmerge@\htool{LMerge}|)}
